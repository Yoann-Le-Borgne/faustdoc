\chapter{Compiling and installing \faust}
\label{install}


The \faust source distribution \lstinline'faust-2.60.3.tar.gz' can be downloaded from GitHub (\myurl{https://github.com/grame-cncm/faust/releases}).

\section{Organization of the distribution}
The first thing is to decompress the downloaded archive. 
\begin{lstlisting}
	tar xzf faust-2.60.3.tar.gz
\end{lstlisting}

The resulting \lstinline'faust-2.60.3/' folder should contain the following elements:

\begin{tabular}{ll}
	\lstinline'architecture/' 		&\faust libraries and architecture files\\
	\lstinline'benchmark'			&tools to measure the efficiency of the generated code\\
	\lstinline'compiler/'			&sources of the \faust compiler\\
	\lstinline'examples/'			&examples of \faust programs\\
	\lstinline'syntax-highlighting/'&	support for syntax highlighting for several editors\\
	\lstinline'documentation/' 		&\faust's documentation, including this manual\\
	\lstinline'tools/'				&tools to produce audio applications and plugins\\
	\lstinline'COPYING'			&license information\\
	\lstinline'Makefile'			&Makefile used to build and install \faust\\
	\lstinline'README'			&instructions on how to build and install \faust
\end{tabular}

\section{Compilation}
\faust has no dependencies outside standard libraries. Therefore the compilation should be straightforward. There is no configuration phase, to compile the \faust compiler simply do :
\begin{lstlisting}
	cd faust-2.60.3/
	make
\end{lstlisting}

If the compilation was successful you can test the compiler before installing it:
\begin{lstlisting}
	[cd faust-2.60.3/]
	./build/bin/faust -v
\end{lstlisting}
It should output:
\begin{lstlisting}
FAUST Version 2.60.3
Embedded backends: 
   DSP to C
   DSP to C++
   DSP to Cmajor
   DSP to CSharp
   DSP to DLang
   DSP to Java
   DSP to JAX
   DSP to Julia
   DSP to old C++
   DSP to Rust
   DSP to WebAssembly (wast/wasm)
Copyright (C) 2002-2023, GRAME - Centre National de Creation Musicale. All rights reserved. 
\end{lstlisting}

Then you can also try to compile one of the examples :
\begin{lstlisting}
	[cd faust-2.60.3/]
	./build/bin/faust examples/generator/noise.dsp
\end{lstlisting}
It should produce some C++ code on the standard output.

\section{Installation}
You can install \faust with:
\begin{lstlisting}
	[cd faust-2.60.3/]
	sudo make install
\end{lstlisting}
or
\begin{lstlisting}
	[cd faust-2.60.3/]
	su
	make install
\end{lstlisting}
depending on your system.

\section{Compilation of the examples}

Once \faust correctly installed, you can have a look at the provided examples in the \lstinline'examples/' folder. 

You can use any of the \lstinline'faust2...' script installed on your system (go in \lstinline'/tools/faust2appls' to get an exhaustive list) to compile the Faust codes available in this folder. For example, if you're a Mac user and you want to turn \lstinline'filtering/vcfWahLab.dsp' into a standalone CoreAudio application with a QT interface, just run:

\lstinline'faust2caqt filtering/vcfWahLab.dsp'

